\documentclass[12pt,a4paper]{article}
\usepackage[UTF8]{ctex}
\usepackage{geometry}
\usepackage{multirow}
\usepackage{array}
\usepackage{graphicx}

\geometry{a4paper,left=20mm,right=20mm,top=20mm,bottom=20mm}
\newcolumntype{P}[1]{>{\raggedright\arraybackslash}p{#1}}

\begin{document}

\begin{center}
    {\heiti % 从这里开始用黑体

{\Huge \textbf{O2O优惠券使用预测}} \\[0.8cm]
{\Huge \textbf{实践报告册}} \\[2.5cm]
  \vspace{4cm}  
    \renewcommand{\arraystretch}{1.8}
\begin{tabular}{|p{4cm}|>{\centering\arraybackslash}p{8cm}|}

        \hline
        \textbf{学年学期:} & 2025-2026学年 第1学期 \\
        \hline
        \textbf{课程名称:} & 机器学习 \\
        \hline
        \textbf{学生学院:} & 智能软件与工程学院 \\
        \hline
        \textbf{专业班级:} &  \\
        \hline
        \textbf{学生学号:} & XXX \\
        \hline
        \textbf{学生姓名:} & XXX \\
        \hline
        \textbf{联系电话:} & XXX \\
        \hline
    \end{tabular}

    } % 黑体到这里结束
\end{center}


\newpage

{\songti

{\songti
\renewcommand{\arraystretch}{1.4}
\begin{center}
\begin{tabular}{|>{\centering\arraybackslash}p{6cm}|>{\centering\arraybackslash}p{8cm}|}
    \hline
    队伍名称和最佳得分 &
    \begin{minipage}[c]{8cm}
        \centering
        截图展示,例如\\[0.5em]
        \includegraphics[width=7cm]{submit.png} % 换成你的图片文件名
    \end{minipage}
    \\ \hline
    成绩提交时间       & 最佳成绩提交时间 \\ \hline
    实践名称           & O2O优惠券使用预测 \\ \hline
\end{tabular}
\end{center}
}

\vspace{0.8cm}

\noindent 一、实践目的和要求

\noindent 完成 O2O 优惠券使用预测阿里天池平台线上测评。

\vspace{0.5cm}

\noindent 二、实践设备

\begin{enumerate}
  \item 操作系统:Windows
  \item 基本硬件配置要求:处理器 Intel i3、内存 8G;
  \item Python 及其组件版本:
  \begin{itemize}
    \item Python 3.6.5 以上
    \item NumPy 1.14.3 以上
    \item Pandas 0.23.0 以上
    \item Scikit-learn 0.19.1 以上
    \item XGBoost 0.6
    \item LightGBM 2.1.0
  \end{itemize}
\end{enumerate}

\vspace{0.5cm}

\noindent 三、实践内容

\noindent 此处由你来填写。实践方法和步骤、实践过程原始记录(数据分析、图表的展示等)、实践结果(部分提交结果的记录)及分析等。


例: 2025.11.25, 数据分析,对xx字段做了xx分析。。。结论。。。。


例: 2025.12.01, 特征工程,做了xx特征,线上AUC达到了。。。


\vspace{0.5cm}

\noindent 四、学习心得

\noindent 此处由你来填写。500 字左右对于此次实践的心得体会。

} % songti 结束
\end{document}
